\documentclass[a4paper]{article}

\usepackage[margin=0.7in]{geometry}
\usepackage[english]{babel}
\usepackage[utf8]{inputenc}
\usepackage{amsmath}
\usepackage{pxfonts}
\usepackage{graphicx}
\usepackage[colorinlistoftodos]{todonotes}
\usepackage{listings,fontspec}
\usepackage{float}
\usepackage{upgreek}
\usepackage{ amssymb }

\lstset{
  basicstyle=\small\ttfamily,
  language= R,
  frame=single,
  numbers=none
}

\lstset{escapeinside={(*@}{@*)}}


\title{Homework 2}

\author{Gitanshu Munjal}

\date{October 28, 2014}

\begin{document}
\maketitle


\section{Question 1}
Write out the full model (equation). State the null and alternative hypotheses in terms of values of parameters (hypothesis above: all RGR's are the same). [20]

\subsection{Full Model(Equation)}

$$Y = \beta_0 +\Updelta\beta_{01} G_1 +\Updelta\beta_{02} G_2 +\Updelta\beta_{03} G_3 +(\beta_1 +\Updelta\beta_{11} G_1 +\Updelta\beta_{12} G_2 +\Updelta\beta_{13} G_3)X + \epsilon$$
Where,
\\ $\beta_0$ and $\beta_1$ are the intercept and slope respectively (for all groups pooled),
\\ $\Updelta\beta_{0i}$ is the effect of group i on the intercept (i=1,2,3; 3 temperature groups), $\sum\Updelta\beta_{0i}=0$
\\ $\Updelta\beta_{1i}$ is the effect of group i on the slope (i=1,2,3; 3 temperature groups), $\sum\Updelta\beta_{1i}=0$
\\ $G_i$ is a "dummy" variable that has a value of 1 for i= 1,2,3 and 0 for all other values of i
\\ $X$ is a continuous variable (with units of time(days) for our model)
\\$\epsilon$ is a random variable with mean 0, variance $\sigma^2$.

\subsection{Hypotheses}
We know that Relative Growth Rate (RGR) is given by slope of the regression line that relates natural logarithm of plant size (size = dry weight for our purposes here) to time. (Evans 1972, Plant growth analysis, p.197). Our null hypothesis ($H_0$) stating that RGR is the same for all groups, independent of temperature can thus be mathematically stated as: 
$$H_0:		\beta_{11} = \beta_{12} = \beta_{13}$$
Given,
$\sum\Updelta\beta_{1i}=0$ and $\beta_{1i}=\beta_1+\Updelta\beta_{1i}$
The null hypothesis can be rewritten as: 
$$H_0:		\Updelta\beta_{11}=\Updelta\beta_{12}=\Updelta\beta_{13}=0$$
\\Where symbols assume their meanings described in section 1.1 above. The alternative hypothesis ($H_1$) can then be stated as:
$$H_1:		at\: least\: one\: of\: the\: slopes(RGRs)\: is\: significantly\: different\: from\: the\: others$$

or equivalently (Given,$\sum\Updelta\beta_{1i}=0$ and $\beta_{1i}=\beta_1+\Updelta\beta_{1i}$)

$$H_1:		at\: least\: one\: of\: the\: group\: effects\: (\Updelta\beta_{1i})\: is\: significantly\: different\: from\: 0$$
\begin{center}
\line(1,0){250}
\end{center}
\pagebreak







\section{Question 2}
Perform a test of this hypothesis by analyzing the data with R, and interpret the corresponding F test. [20]

\subsection{Test of Hypothesis: Code and Relevant Output}

\begin{lstlisting}[language=R,frame=single,numbers=none]
library(multcomp)
library(car)

clover<- read.table("C:\\Users\\gitanshu\\Desktop\\Clover.dat",header=T,sep=",")

clover$group<-as.factor(clover$group)
options(contrasts=c("contr.sum", "contr.poly"))

model1 <- lm(lnwt~group*days, data=clover)
summary(model1)
\end{lstlisting}

\begin{lstlisting}[language=R,frame=none]
Call:
lm(formula = lnwt ~ group * days, data = clover)

Residuals:
    Min      1Q  Median      3Q     Max 
-0.2851 -0.1037 -0.0346  0.1613  0.2764 

Coefficients:
             Estimate Std. Error t value Pr(>|t|)    
(Intercept)  1.583504   0.062411  25.372  < 2e-16 ***
group1      -0.089557   0.088364  -1.014    0.317    
group2      -0.048508   0.090520  -0.536    0.595    
days         0.091338   0.002540  35.955  < 2e-16 ***
group1:days -0.022139   0.003585  -6.176 2.95e-07 ***
group2:days  0.002504   0.003655   0.685    0.497    
---
Signif. codes:  0 �***� 0.001 �**� 0.01 �*� 0.05 �.� 0.1 � � 1

Residual standard error: 0.1705 on 39 degrees of freedom
Multiple R-squared:  0.9769,	Adjusted R-squared:  0.9739 
(*@ \textcolor{blue}{F-statistic: 329.2 on 5 and 39 DF,  p-value: < 2.2e-16} @*)
\end{lstlisting}

Above is the result of the correct F-test for our overall model  as it uses 5 degrees of freedom for the model and 39 degrees of freedom for the error term. A \textcolor{blue}{highly significant} p-value here (p-value: < 2.2e-16) at the 95\% confidence level encourages us to partition the model in order to better understand the source of significance.
\begin{lstlisting}[language=R]
anova(model1)
\end{lstlisting}

\begin{lstlisting}[language=R,frame=none]
Analysis of Variance Table

Response: lnwt
           Df Sum Sq Mean Sq  F value    Pr(>F)    
group       2  8.549   4.275  146.988 < 2.2e-16 ***
days        1 37.953  37.953 1305.062 < 2.2e-16 ***
(*@ \textcolor{blue}{group:days}@*)  2  1.368   0.684   23.519  (*@ \textcolor{blue}{1.992e-07 ***}@*)
Residuals  39  1.134   0.029                       
---
Signif. codes:  0 �***� 0.001 �**� 0.01 �*� 0.05 �.� 0.1 � � 1\end{lstlisting}
Above are the results of sequential (Type 1) tests for our model. A highly \textcolor{blue}{significant interaction term} (p-value: 1.992e-07) 
here for the interaction between temperature groups and days indicates that the RGRs for the temperature groups are \textcolor{blue}{NOT} all the same and so we \textcolor{blue}{reject our null hypothesis} from Question 2 at the 95\% confidence level.
\begin{center}
\line(1,0){250}
\end{center}
\pagebreak







\section{Question 3}
Determine which group, if any, differs in RGR. Use the Bonferroni correction. [20]

\subsection{Simultaneous test of differences between relative growth rates (slopes)}
\begin{lstlisting}[language=R]
b11<-as.vector(cbind(0,0,0,1,1,0))
b12<-as.vector(cbind(0,0,0,1,0,1))
b13<-as.vector(cbind(0,0,0,1,-1,-1))

H01s <- rbind(B12_B11 =   b12-b11,
              B13_B11 =   b13-b11,
              B13_B12 =   b13-b12)
tH01s <- glht(model1, linfct = H01s)
summary(tH01s, test = adjusted("bonferroni"))
\end{lstlisting}
\begin{lstlisting}[frame=none,language=R]
Simultaneous Tests for General Linear Hypotheses

Fit: lm(formula = lnwt ~ group * days, data = clover)

Linear Hypotheses:
             Estimate Std. Error t value Pr(>|t|)    
B12_B11 == 0 0.024643   0.006317   3.901   0.0011 (*@ \textcolor{blue}{**}@*) 
B13_B11 == 0 0.041773   0.006113   6.834 1.09e-07 (*@ \textcolor{blue}{***}@*)
B13_B12 == 0 0.017131   0.006236   2.747   0.0272 (*@ \textcolor{blue}{*}@*)  
---
Signif. codes:  0 �***� 0.001 �**� 0.01 �*� 0.05 �.� 0.1 � � 1
(Adjusted p values reported -- bonferroni method)
\end{lstlisting}
The results from the above simultaneous tests of hypotheses (controlled for familywise error rate using the Bonferroni method) indicate signicant differences in RGR between all groups as all comparisons were found to have a significant p value ($\alpha_{critical}=\frac{\alpha}{3}=\frac{0.05}{3}=0.0167$). In other words, the RGR for each group is significantly different from the RGRs for all other groups at the 95\% confidence level.

\begin{center}
\line(1,0){250}
\end{center}




\section{Question 4}
Determine if the weight of plants grown in the lowest temperature for 27 days is significantly different from the weight of plants grown at the highest temperature for 11 days: First, (a) state Ho and Ha in words and with equations, or using statements about parameter values; then, (b) to test this hypothesis, build an L vector and use it in a linear hypothesis in R, and state your conclusion. [20]

\subsection{Hypotheses}
Let $Y_{1.27}$ denote the weight of plants grown at the lowest temperature (group 1) for 27 days and let $Y_{3.11}$ denote the weight of plants grown at the highest temperature (group 3) for 11 days. Then the null hypothesis is that there is no significant difference between the weights of group 1 plants grown for 27 days in comparison to the weights of group 3 plants grown for 11 days. Mathematically, this can be stated as:
$$H_0: Y_{1.27}=Y_{3.11}$$
or equivalently, $$H_0: Y_{1.27}-Y_{3.11}=0$$

The alternative hypothesis would then be that there is indeed a significant difference between the weights of group 1 plants grown for 27 days in comparison to the weights of group 3 plants grown for 11 days. Mathematically, this can be stated as::
$$H_0: Y_{1.27}\neq Y_{3.11}$$
or equivalently, $$H_0: Y_{1.27}-Y_{3.11}\neq 0$$

Then using the model matrix and our previously defined full model (with described restrictions), we can calculate L vectors for $Y_{1.27}$ and $Y_{3.11}$. The L vector for the equivalent null hypothesis can then be calculated simply by subtracting the two previously calculated vectors.

\subsection{Test of Hypothesis: Code and Relevant Output}
\begin{lstlisting}[language=R]
Y1.27<-as.vector(cbind(1,1,0,27,27,0))
Y3.11<-as.vector(cbind(1,-1,-1,11,-11,-11))


H0s <- rbind(Hypothesis=Y1.27-Y3.11) 

tH0s <- glht(model1, linfct = H0s)
summary(tH0s, test = univariate())
\end{lstlisting}

\begin{lstlisting}[language=R,frame=none]
	 Simultaneous Tests for General Linear Hypotheses

Fit: lm(formula = lnwt ~ group * days, data = clover)

Linear Hypotheses:
                Estimate Std. Error t value Pr(>|t|)    
Hypothesis == 0  0.42006    0.07932   5.296 4.91e-06 ***
---
Signif. codes:  0 �***� 0.001 �**� 0.01 �*� 0.05 �.� 0.1 � � 1
(Univariate p values reported)
\end{lstlisting}
The results from the above test of hypothesis (not controlled for familywise error rate because we are only interested in one test) indicate that there is indeed a signficant differnece (p-value=4.91e-06) at the 95\% confidence level between the weight of plants grown in the lowest temperature for 27 days and the weight of plants grown at the highest temperature for 11 days.

\begin{center}
\line(1,0){250}
\end{center}



\section{Question 5}
Why would the above analysis be incorrect if the same 3 plants per temperature treatment had been used throughout the experiment as depicted below? What typical assumption of regression would probably be violated? Why? Assume that the plants are not destroyed when measured (if you have trouble assuming this, imagine that you are measuring height instead of mass). Assume that the plants are not in the same greenhouses. [20]
\subsection{Violation of Independence of Error}
The analysis would be incorrect because the independence of error assumption would be violated and so our usual tests might not hold. One of the most important assumptions of the tests we used in our analysis  was independence of errors which we achieved in the past by having true replicates. By making measurements on the same 3 plants over time we are introducing a repeated measures type design and violating the independence of errors assumption adding some structure to the error (consider 1 of the 3 plants, the error for the second measurement would be related to the first measurement as the experimental unit is the same, so on and so forth, we cannot consider them as true replicates.)
\begin{center}
\line(1,0){250}
\end{center}
\end{document}

